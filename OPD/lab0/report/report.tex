\documentclass[12pt,onecolumn]{article}

\usepackage{listings}
\usepackage{float}
\usepackage{mathtools}
\usepackage[russian]{babel}
\everymath{\displaystyle}

\usepackage[usenames]{color}
\usepackage{colortbl}

\usepackage{geometry}
\usepackage{minted}
\geometry{
  a4paper,
  top=15mm, 
  right=10mm, 
  bottom=15mm, 
  left=10mm
}

\definecolor{dkgreen}{rgb}{0,0.6,0}
\definecolor{gray}{rgb}{0.5,0.5,0.5}
\definecolor{mauve}{rgb}{0.58,0,0.82}

\lstset{frame=tb,
  language=sh,
    aboveskip=3mm,
      belowskip=3mm,
        showstringspaces=false,
	  columns=flexible,
	    basicstyle={\small\ttfamily},
	      numbers=none,
	        numberstyle=\tiny\color{gray},
		  keywordstyle=\color{blue},
		    commentstyle=\color{dkgreen},
		      stringstyle=\color{mauve},
		        breaklines=true,
			  breakatwhitespace=true,
			    tabsize=3
			    }
\begin{document}

\begin{center}
    Федеральное государственное автономное образовательное учреждение высшего образования\\
	«Национальный исследовательский университет ИТМО»
\end{center}
\vspace{1cm}


\begin{center}
    \large \textbf{Отчет}\\
    \textbf{по лабораторной работе №1}\\
    \large \textbf{«Основные команды ОС семейства UNIX»}\\
     по дисциплине «Основы профессиональной деятельности»\\
	\vspace{1cm}
    Вариант №1806\\
\end{center}

\vspace{10cm}
\begin{flushright}
  Выполнил: Кокорин Всеволод Вячеславович, группа P3118\\
  Преподаватель:Ткешелашвили Н.М.\\
\end{flushright}

\vspace{5cm}
\begin{center}
    г. Санкт-Петербург\\
    2022г.
\end{center}
\newpage
\tableofcontents
\newpage
\section{Задание 1}
Создать приведенное в варианте дерево каталогов и файлов с содержимым. В качестве корня дерева использовать каталог lab0 своего домашнего каталога. Для создания и навигации по дереву использовать команды: mkdir, echo, cat, touch, ls, pwd, cd, more, cp, rm, rmdir, mv.\\ 
\includegraphics[width=15cm]{img/1-1.png}
\newpage
\inputminted{sh}{code/1}
\newpage
\section{Задание 2}
Установить согласно заданию права на файлы и каталоги при помощи команды chmod, используя различные способы указания прав.\\
\inputminted{sh}{code/2.sh}
\newpage
\section{Задание 3}
Скопировать часть дерева и создать ссылки внутри дерева согласно заданию при помощи команд cp и ln, а также комманды cat и перенаправления ввода-вывода.
\inputminted{sh}{code/3.sh}
\newpage
\section{Итоговое дерево файлов, после выполнения 3 задания}
\VerbatimInput{code/3r.txt}
\newpage
\section{Задание 4}
\inputminted{sh}{code/4.sh}
\newpage
\section{Заданние 5}
\inputminted{sh}{code/5.sh}
\newpage
\section{Вывод}
По ходу выполнения работы,  я ознакомился со способами взаимодействия с OC UNIX, изучил основные команды, для листинга, фильтрации, создание/удаления файлов, назначения прав доступа.
\end{document}
