\documentclass[12pt,onecolumn]{article}
\usepackage{float}
\usepackage{mathtools}
\usepackage[russian]{babel}
\everymath{\displaystyle}

\usepackage[usenames]{color}
\usepackage{colortbl}

\usepackage{geometry}
\geometry{
  a4paper,
  top=25mm, 
  right=30mm, 
  bottom=25mm, 
  left=30mm
}

\begin{document}

\begin{center}
    Федеральное государственное автономное образовательное учреждение\\ высшего образования\\
«Национальная научно-образовательная корпорация ИТМО»\\
Факультет Программной Инженерии и Компьютерной Техники \\
\end{center}
\vspace{1cm}


\begin{center}
    \large \textbf{Вариант №}\\
    \textbf{Лабораторная работа №}\\
    по дисциплине\\
    \textbf{\textcolor{red}{\textit{'программирование'}}}
\end{center}

\vspace{2cm}

\begin{flushright}
  Выполнил Студент  группы P3118\\
  \textbf{Кокорин Всеволод Вячеславович}\\
  Преподаватель: \\
  \textbf{Письмак Алексей Евгеньевич}\\
\end{flushright}

\vspace{6cm}
\begin{center}
    г. Санкт-Петербург\\
    2022г.
\end{center}

\newpage
\section{Текст задания}

\section{Исходный код программы}
https://github.com/...

\section{Результат выполнения}

\section{Вывод}



\end{document}
